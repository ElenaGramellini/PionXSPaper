%% ****** Start of file template.aps ****** %
%%
%%
%%   This file is part of the APS files in the REVTeX 4 distribution.
%%   Version 4.0 of REVTeX, August 2001
%%
%%
%%   Copyright (c) 2001 The American Physical Society.
%%
%%   See the REVTeX 4 README file for restrictions and more information.
%%
%
% This is a template for producing manuscripts for use with REVTEX 4.0
% Copy this file to another name and then work on that file.
% That way, you always have this original template file to use.
%
% Group addresses by affiliation; use superscriptaddress for long
% author lists, or if there are many overlapping affiliations.
% For Phys. Rev. appearance, change preprint to twocolumn.
% Choose pra, prb, prc, prd, pre, prl, prstab, or rmp for journal
%  Add 'draft' option to mark overfull boxes with black boxes
%  Add 'showpacs' option to make PACS codes appear
\documentclass[aps,prl,twocolumn,showpacs,superscriptaddress,groupedaddress]{revtex4}  % for review and submission
%\documentclass[aps,preprint,showpacs,superscriptaddress,groupedaddress]{revtex4}  % for double-spaced preprint
\usepackage{graphicx}  % needed for figures
\usepackage{dcolumn}   % needed for some tables
\usepackage{bm}        % for math
\usepackage{amssymb}   % for math
 \usepackage{comment} % for large comment sections
\usepackage{color}


% avoids incorrect hyphenation, added Nov/08 by SSR
\hyphenation{ALPGEN}
\hyphenation{EVTGEN}
\hyphenation{PYTHIA}


\begin{document}

% The following information is for internal review, please remove them for submission
\widetext
\leftline{Version 0 as of \today}
\leftline{Primary authors: Elena Gramellini}
\leftline{To be submitted to PRD.}
\leftline{Comment to {\tt elenag@fnal.gov} by xxx, yyy}
\centerline{\em LArIAT INTERNAL DOCUMENT -- NOT FOR PUBLIC DISTRIBUTION}

% the following line is for submission, including submission to the arXiv!!
%\hspace{5.2in} \mbox{Fermilab-Pub-04/xxx-E}

\title{Measurement of the ($\pi^-$, Ar) total hadronic cross section at the LArIAT experiment}
\input lariat_author_list.tex       % LArIAT authors (remove the first 3 lines
                                       % of this file prior to submission, they
                                       % contain a time stamp for the authorlist)
                                       % (includes institutions and visitors)
\date{\today}


\begin{abstract}
We present the measurement of the negative pion total hadronic cross section on Argon in the 100-1050 MeV kinetic energy range as performed at the LArIAT experiment.   Deploying a Liquid Argon Time Projection Chamber (LArTPC) in a dedicated calibration test beam line at Fermilab, LArIAT (Liquid Argon In A Testbeam) aims to experimentally calibrate this technology in a controlled environment. We use LArIAT's beamline detectors to identify candidate pions and measure their kinetic energy prior to entering the LArTPC. Once the pion candidate reaches the LAr active volume, we apply the ``thin slice method", a new technique to measure hadron-argon cross sections based on the combination of tracking and calorimetric capability of the LArTPC technology. 
Albeit our measurement of the  ($\pi^-$-Ar) total hadronic cross section is in general agreement with the predictions by the Geant4 Bertini Cascade model derived from historic measurements on lighter and heavier nuclei, a hint to a shape difference is present.
The ($\pi^-$, Ar) total cross-section has never been measured before; the outcome of this measurement will enable to quantify and reduce the systematic associated with the hadronic interaction models in neutrino-argon interactions in current ant future LAr experiments such as the Short Baseline Neutrino and DUNE.
 



\end{abstract}

\pacs{13.15.+g Neutrino interactions, 14.40.-n	Mesons,  14.40.Be	Light mesons (S=C=B=0), 13.75.-n	Hadron-induced low- and intermediate-energy reactions and scattering (energy $le$ 10 GeV)}
\maketitle

\section{\label{sec:Motivations}Motivations}
Motivations

\section{\label{sec:ExperimentalSetup}Experimental Setup}
Setup
\subsection{\label{sec:Beamline}Beamline} 
Beamline
\subsection{\label{sec:LArTPC}LArTPC}
LArTPC
\textcolor{red}{At this point, I should have explained how I get my pool of candidates.}

\section{\label{sec:ThinSliceMethod}Thin Slice Method}
Once we have selected the 40841 beamline pion candidates and we have identified the TPC corresponding track, we apply the thin slice method to measure the cross section. 

Historic measurements of hadronic cross sections are performed on thin targets [\textcolor{red}{A CIT is needed here}]. These experiments consists in shooting a beam of particles with a known flux on a thin slab of material and recording the outgoing flux, under the assumption that the target centers are uniformly distributed in the material and  that no center of interaction sits in front of another for what it concerns the process under investigation. In this thin target approximation, the ratio between the number of particles interacting in the target $N_{\text{Int}}$ and the number of incident particles $N_{\text{Inc}}$ on the target estimates the interaction probability $P_{\text{Int}}$, which is the complementary to one of the survival probability $P_{\text{Surv}}$. 
Equation \ref{eq:thinTargetXS} 
\begin{equation}
P_{\text{Surv}} = 1- P_{\text{Int}}  = 1 - \frac{N_{\text{Int}}}{N_{\text{Inc}}} = e^{-\sigma_{\text{TOT}}\text{ } n \text{ }\delta X}
\label{eq:thinTargetXS}
\end{equation}
describes the probability for a particle to survive the thin target; this formula relates  the interaction probability to the total hadronic cross section, $\sigma_{\text{TOT}}$, the density of the target centers, $n$,  and  the thickness of the target  along the incident hadron direction, $\delta X$.  Since the target is thin compared to the interaction length of the process considered,  a simple proportionality relationship between the cross section and the interaction probabily, $\frac{N_{\text{Int}}}{N_{\text{Inc}}}$, is found by Taylor expanding the exponential function:
 \begin{equation}
 \sigma_{\text{TOT}}  = \frac{1}{n \text{ }\delta X}\frac{N_{\text{Int}}}{N_{\text{Inc}}}.
\label{eq:thinTargetXSSolved}
\end{equation}

Since the interaction length of pions in liquid argon is expected to be of the order of 50 cm [\textcolor{red}{A CIT is needed here}], the LArIAT TPC, with its 90 cm of length, is not a thin target. However, the granularity of the LArIAT LArTPC allows us to assess the presence of a pion and to measure its kinetic energy approximately every 4.7~mm along its trajectory in the detector's active volume. We can thus treat the argon volume as a sequence of many adjacent thin targets, recovering the thin target approximation in each slab of argon. 

As described in Chapter \ref{sec:experimentDescription}, LArIAT induction and collection planes consist of 240 wires each at 4 mm spacing. The wires are oriented at +/- $60^{\circ}$ from the vertical direction, while the beam direction is oriented 3 degrees off the $z$ axis in the $XZ$ plane.  The collection wires collect signals proportional to the energy deposited by the hadron along its path in a  $\delta${\emph{X}} = 4 mm/(sin($60^{\circ}$)cos($3^{\circ}$)) $\approx$ 4.7~mm slab of liquid argon. Thus, one can think to slice the TPC into many thin targets of $\delta${\emph{X}} = 4.7~mm thickness along the direction of the incident particle, making a measurement at each wire along the path, as sketched in Figure \ref{fig:TPCGran}.

\begin{figure}
  \centering  
%\includegraphics[width=0.8\textwidth]{Chapter-4/Images/LArSlice.png}
\caption{Representation of sliced LAr Volume.}
\label{fig:TPCGran}
\end{figure}


Considering each slice {\emph{j}}  a ``thin target",  we can apply the cross section calculation from Equation~\ref{eq:thinTargetXSSolved} iteratively, evaluating the kinetic energy of the hadron as it enters each slice, $E_{j}^{kin}$.  For each WC2TPC matched particle, the energy of the hadron entering the TPC is known thanks to the momentum and mass determination by the tertiary beamline, 

\begin{equation}
 E^{kin}_{Front Face}  = \sqrt{p^2_{Beam} - m^2_{Beam}} - m_{Beam} - E_{loss},
\label{eq:enFF}
\end{equation}
where $E_{loss}$ is a correction for the kinetic energy loss in the uninstrumented material between the beamline and the TPC front face. While propagating through the target,  the kinetic energy of the hadron at each slab is determined by subtracting the energy deposited by the particle in the previous slabs. For example, at the $j^{th}$ slab of a track, the kinetic energy will be

\begin{equation}
 E_{j}^{kin} =  E^{kin}_{Front Face} - \sum_{i < j} E_{\text{Dep},i},
\label{eq:KEj}
\end{equation}
where $E_{\text{Dep},i}$ is the energy deposited at each argon slice before the $j^{th}$ point as measured by the calorimetry associated with the tracking.


If the particle enters a slice, it contributes to the $N_{\text{Inc}}( E^{kin})$ distribution in the energy bin corresponding to its kinetic energy in that slice.  Within the slice, the hadron may or may not interact. If it interacts in the slice, it  contributes also to the $N_{\text{Int}}(E^{kin})$ distribution in the appropriate energy bin; this occurrence corresponds to the end of the hadron tracking. If the hadron does not interact, it will enter the next slice and the interaction evaluation starts again.
The process is applied to all the hadrons in the sample; the cross section as a function of kinetic energy, $\sigma_{TOT}( E^{kin})$ is then evaluated to be proportional to the ratio $\frac{N_{\text{Int}}( E^{kin})}{N_{\text{Inc}}( E^{kin})}$ -- bin by bin ratio. 


Our goal is to measure the total interaction cross section, independently  from the topology of the interaction. Thus, we determine that a hadron interacted simply by requiring that the last point of the WC2TPC matched track lies in a slice within the fiducial volume, whose boundaries are defined in Table \ref{tab:FidVol}. If the TPC track ends within the fiducial volume, its last point will be the interaction point; if the track crosses the boundaries of the fiducial volume, the track will be considered ``through going" and no interaction point will be found. The only points of the hadronic candidate track considered to fill the  $N_{\text{Int}}$ and  $N_{\text{Inc}}$ distributions are the ones contained in the fiducial volume. 
 
 A notable background pertinent only to the $N_{\text{Int}}$  distribution are cases in which the hadrons decays inside the TPC. In those cases in fact, the tracking ends inside the TPC but the interaction is not hadronic. The handling of decay background is treated in a slightly different way for the pion and kaon section, details can be found in sections \ref{ch:PionXSBkgSub} and \ref{ch:KaonXSRaw} respectively.



\begin{table}[t]
\centering
\begin{tabular}{|l|r|r|}
\hline
& min   &  max  \\ \hline
$X$ & 1 cm   & 46 cm  \\ \hline
$Y$ & -15 cm   & 15  cm  \\ \hline
$Z$ & 0 cm   & 86 cm  \\ \hline
\end{tabular}
\caption{Fiducial volume boundaries used to determine cross section interaction point. }
\label{tab:FidVol}
\end{table}


\subsection{\label{sec:RawXS}Raw Cross Section}
\subsection{\label{sec:Corrections}Corrections}

\section{\label{sec:Results}Results}
% sections are not used for PRL papers
Results shown in Fig.~\ref{fig:epsart}.

\begin{figure}
\includegraphics[width =0.4\textwidth ]{TheRealMoneyPlot}
\caption{\label{fig:epsart} Top: ($\pi^-$-Ar) total hadronic cross section for  scattering angles greater than 5$^\circ$ measured in the combined sample, statistical uncertainty and systematic uncertainty in black. The Geant4 prediction for the total hadronic cross section for angle scattering greater than 5$^\circ$ is displayed in green. Bottom: Relative difference between the measured cross section and the Geant4 prediction. }
\end{figure}


\begin{comment}
This sample document demonstrates proper use of REV\TeX~4 (and
\LaTeXe) in mansucripts prepared for submission to APS
journals. Further information can be found in the REV\TeX~4
documentation included in the distribution or available at
\url{http://publish.aps.org/revtex4/}.

When commands are referred to in this example file, they are always
shown with their required arguments, using normal \TeX{} format. In
this format, \verb+#1+, \verb+#2+, etc. stand for required
author-supplied arguments to commands. For example, in
\verb+\section{#1}+ the \verb+#1+ stands for the title text of the
author's section heading, and in \verb+\title{#1}+ the \verb+#1+
stands for the title text of the paper.

Line breaks in section headings at all levels can be introduced using
\textbackslash\textbackslash. A blank input line tells \TeX\ that the
paragraph has ended. Note that top-level section headings are
automatically uppercased. If a specific letter or word should appear in
lowercase instead, you must escape it using \verb+\lowercase{#1}+ as
in the word ``via'' above.

%\subsection{\label{sec:level2}Second-level heading: Formatting}
% subsections are not used for PRL papers

This file may be formatted in both the \texttt{preprint} and
\texttt{twocolumn} styles. \texttt{twocolumn} format may be used to
mimic final journal output. Either format may be used for submission
purposes; however, for peer review and production, APS will format the
article using the \texttt{preprint} class option. Hence, it is
essential that authors check that their manuscripts format acceptably
under \texttt{preprint}. Manuscripts submitted to APS that do not
format correctly under the \texttt{preprint} option may be delayed in
both the editorial and production processes.

The \texttt{widetext} environment will make the text the width of the
full page.  The width-changing commands only take effect in \texttt{twocolumn}
formatting. It has no effect if \texttt{preprint} formatting is chosen
instead.

To cite bibliography entries, use the \verb+\cite{#1}+ command. Most
journal styles will display the corresponding number(s) in square
brackets: \cite{d0det}. To avoid the square brackets, use
\verb+\onlinecite{#1}+: Refs.~\onlinecite{d0det} and
\onlinecite{geant,pythia}. REV\TeX\ ``collapses'' lists of
consecutive reference numbers where possible. We now cite everyone
together \cite{geant, pythia, cteq}, and once again
(Refs.~\onlinecite{geant, pythia, cteq}). Note that the references
were also sorted into the correct numerical order as well.

Footnotes are produced using the \verb+\footnote{#1}+ command. Most
APS journal styles put footnotes into the bibliography. REV\TeX~4 does
this as well, but instead of interleaving the footnotes with the
references, they are listed at the end of the references. Because the correct
numbering of the footnotes must occur after the numbering of the
references, an extra pass of \LaTeX\ is required in order to get the
numbering correct.


Inline math may be typeset using the \verb+$+ delimiters. Bold math
symbols may be achieved using the \verb+bm+ package and the
\verb+\bm{#1}+ command it supplies. For instance, a bold $\alpha$ can
be typeset as \verb+$\bm{\alpha}$+ giving $\bm{\alpha}$. Fraktur and
Blackboard (or open face or double struck) characters should be
typeset using the \verb+\mathfrak{#1}+ and \verb+\mathbb{#1}+ commands
respectively. Both are supplied by the \texttt{amssymb} package. For
example, \verb+$\mathbb{R}$+ gives $\mathbb{R}$ and
\verb+$\mathfrak{G}$+ gives $\mathfrak{G}$

In \LaTeX\ there are many different ways to display equations, and a
few preferred ways are noted below. Displayed math will center by
default. Use the class option \verb+fleqn+ to flush equations left.

Below we have numbered single-line equations; this is the most common
type of equation in \textit{Physical Review}:
\begin{eqnarray}
\chi_+(p)\alt{\bf [}2|{\bf p}|(|{\bf p}|+p_z){\bf ]}^{-1/2}
\left(
\begin{array}{c}
|{\bf p}|+p_z\\
px+ip_y
\end{array}\right)\;,
\\
\left\{%
 \openone234567890abc123\alpha\beta\gamma\delta1234556\alpha\beta
 \frac{1\sum^{a}_{b}}{A^2}%
\right\}%
\label{eq:one}.
\end{eqnarray}
Note the open one in Eq.~(\ref{eq:one}).

Not all numbered equations will fit within a narrow column this
way. The equation number will move down automatically if it cannot fit
on the same line with a one-line equation:
\begin{equation}
\left\{
 ab12345678abc123456abcdef\alpha\beta\gamma\delta1234556\alpha\beta
 \frac{1\sum^{a}_{b}}{A^2}%
\right\}.
\end{equation}

When the \verb+\label{#1}+ command is used [cf. input for
Eq.~(\ref{eq:one})], the equation can be referred to in text without
knowing the equation number that \TeX\ will assign to it. Just
use \verb+\ref{#1}+, where \verb+#1+ is the same name that used in
the \verb+\label{#1}+ command.

Unnumbered single-line equations can be typeset
using the \verb+\[+, \verb+\]+ format:
\[g^+g^+ \rightarrow g^+g^+g^+g^+ \dots ~,~~q^+q^+\rightarrow
q^+g^+g^+ \dots ~. \]


Figures may be inserted by using either the \texttt{graphics} or
\texttt{graphicx} packages. These packages both define the
\verb+\includegraphics{#1}+ command, but they differ in how optional
arguments for specifying the orientation, scaling, and translation of the
figure. Fig.~\ref{fig:epsart} shows a figure that is small enough to
fit in a single column. It is embedded using the \texttt{figure}
environment which provides both the caption and the imports the figure
file.

\begin{figure}
\includegraphics[scale=0.8]{TheRealMoneyPlot.ps}
\caption{\label{fig:epsart} A figure caption. The figure captions are
automatically numbered.}
\end{figure}

Fig.~\ref{fig:wide} is a figure that is too wide for a single column,
so instead the \texttt{figure*} environment has been used.
\begin{figure*}
%\includegraphics{fig_2.ps}% Here is how to import EPS art
\caption{\label{fig:wide}Use the figure* environment to get a wide
figure that spans the page in \texttt{twocolumn} formatting.}
\end{figure*}


The heart of any table is the \texttt{tabular} environment which gives
the rows of the tables. Each row consists of column entries separated
by \verb+&+'s and terminates with \textbackslash\textbackslash. The
required argument for the \texttt{tabular} environment
specifies how data are displayed in the columns. For instance, entries
may be centered, left-justified, right-justified, aligned on a decimal
point. Extra column-spacing may be be specified as well, although
REV\TeX~4 sets this spacing so that the columns fill the width of the
table. Horizontal rules are typeset using the \verb+\hline+
command. The doubled (or Scotch) rules that appear at the top and
bottom of a table can be achieved enclosing the \texttt{tabular}
environment within a \texttt{ruledtabular} environment. Rows whose
columns span multiple columns can be typeset using the
\verb+\multicolumn{#1}{#2}{#3}+ command (for example, see the first
row of Table~\ref{tab:table3}).

Tables~\ref{tab:table1}-\ref{tab:table4} show various effects. Tables
that fit in a narrow column are contained in a \texttt{table}
environment. Table~\ref{tab:table3} is a wide table set with the
\texttt{table*} environment. Long tables may need to break across
pages. The most straightforward way to accomplish this is to specify
the \verb+[H]+ float placement on the \texttt{table} or
\texttt{table*} environment. However, the standard \LaTeXe\ package
\texttt{longtable} will give more control over how tables break and
will allow headers and footers to be specified for each page of the
table. A simple example of the use of \texttt{longtable} can be found
in the file \texttt{summary.tex} that is included with the REV\TeX~4
distribution.

There are two methods for setting footnotes within a table (these
footnotes will be displayed directly below the table rather than at
the bottom of the page or in the bibliography). The easiest
and preferred method is just to use the \verb+\footnote{#1}+
command. This will automatically enumerate the footnotes with
lowercase roman letters. However, it is sometimes necessary to have
multiple entries in the table share the same footnote. In this case,
there is no choice but to manually create the footnotes using
\verb+\footnotemark[#1]+ and \verb+\footnotetext[#1]{#2}+.
\texttt{\#1} is a numeric value. Each time the same value for
\texttt{\#1} is used, the same mark is produced in the table. The
\verb+\footnotetext[#1]{#2}+ commands are placed after the \texttt{tabular}
environment. Examine the \LaTeX\ source and output for
Tables~\ref{tab:table1} and \ref{tab:table2} for examples.

\begin{table}
\caption{\label{tab:table1}This is a narrow table which fits into a
narrow column when using \texttt{twocolumn} formatting. Note that
REV\TeX~4 adjusts the intercolumn spacing so that the table fills the
entire width of the column. Table captions are numbered
automatically. This table illustrates left-aligned, centered, and
right-aligned columns.  }
\begin{ruledtabular}
\begin{tabular}{lcr}
Left\footnote{Note a.}&Centered\footnote{Note b.}&Right\\
\hline
1 & 2 & 3\\
10 & 20 & 30\\
100 & 200 & 300\\
\end{tabular}
\end{ruledtabular}
\end{table}

\begin{table}
\caption{\label{tab:table2}A table with more columns still fits
properly in a column. Note that several entries share the same
footnote. Inspect the \LaTeX\ input for this table to see
exactly how it is done.}
\begin{ruledtabular}
\begin{tabular}{cccccccc}
 &$r_c$ (\AA)&$r_0$ (\AA)&$\kappa r_0$&
 &$r_c$ (\AA) &$r_0$ (\AA)&$\kappa r_0$\\
\hline
Cu& 0.800 & 14.10 & 2.550 &Sn\footnotemark[1]
& 0.680 & 1.870 & 3.700 \\
Ag& 0.990 & 15.90 & 2.710 &Pb\footnotemark[2]
& 0.450 & 1.930 & 3.760 \\
Au& 1.150 & 15.90 & 2.710 &Ca\footnotemark[3]
& 0.750 & 2.170 & 3.560 \\
Mg& 0.490 & 17.60 & 3.200 &Sr\footnotemark[4]
& 0.900 & 2.370 & 3.720 \\
Zn& 0.300 & 15.20 & 2.970 &Li\footnotemark[2]
& 0.380 & 1.730 & 2.830 \\
Cd& 0.530 & 17.10 & 3.160 &Na\footnotemark[5]
& 0.760 & 2.110 & 3.120 \\
Hg& 0.550 & 17.80 & 3.220 &K\footnotemark[5]
&  1.120 & 2.620 & 3.480 \\
Al& 0.230 & 15.80 & 3.240 &Rb\footnotemark[3]
& 1.330 & 2.800 & 3.590 \\
Ga& 0.310 & 16.70 & 3.330 &Cs\footnotemark[4]
& 1.420 & 3.030 & 3.740 \\
In& 0.460 & 18.40 & 3.500 &Ba\footnotemark[5]
& 0.960 & 2.460 & 3.780 \\
Tl& 0.480 & 18.90 & 3.550 & & & & \\
\end{tabular}
\end{ruledtabular}
\footnotetext[1]{Here's the first, from Ref.~\onlinecite{pdg}.}
\footnotetext[2]{Here's the second.}
\footnotetext[3]{Here's the third.}
\footnotetext[4]{Here's the fourth.}
\footnotetext[5]{And etc.}
\end{table}

\begin{table*}
\caption{\label{tab:table3}This is a wide table that spans the page
width in \texttt{twocolumn} mode. It is formatted using the
\texttt{table*} environment. It also demonstates the use of
\textbackslash\texttt{multicolumn} in rows with entries that span
more than one column.}
\begin{ruledtabular}
\begin{tabular}{ccccc}
 &\multicolumn{2}{c}{$D_{4h}^1$}&\multicolumn{2}{c}{$D_{4h}^5$}\\
 Ion&1st alternative&2nd alternative&lst alternative
&2nd alternative\\ \hline
 K&$(2e)+(2f)$&$(4i)$ &$(2c)+(2d)$&$(4f)$ \\
 Mn&$(2g)$\footnote{The $z$ parameter of these positions is $z\sim\frac{1}{4}$.}
 &$(a)+(b)+(c)+(d)$&$(4e)$&$(2a)+(2b)$\\
 Cl&$(a)+(b)+(c)+(d)$&$(2g)$\footnotemark[1]
 &$(4e)^{\text{a}}$\\
 He&$(8r)^{\text{a}}$&$(4j)^{\text{a}}$&$(4g)^{\text{a}}$\\
 Ag& &$(4k)^{\text{a}}$& &$(4h)^{\text{a}}$\\
\end{tabular}
\end{ruledtabular}
\end{table*}

\begin{table}
\caption{\label{tab:table4}Numbers in columns Three--Five have been
aligned by using the ``d'' column specifier (requires the
\texttt{dcolumn} package). Non-numeric entries (those entries without
a ``.'') in a ``d'' column are aligned on the decimal point. Use the
``D'' specifier for more complex layouts. }
\begin{ruledtabular}
\begin{tabular}{ccddd}
One&Two&\mbox{Three}&\mbox{Four}&\mbox{Five}\\
\hline
one&two&\mbox{three}&\mbox{four}&\mbox{five}\\
He&2& 2.77234 & 45672. & 0.69 \\
C\footnote{Some tables require footnotes.}
  &C\footnote{Some tables need more than one footnote.}
  & 12537.64 & 37.66345 & 86.37 \\
\end{tabular}
\end{ruledtabular}
\end{table}



\textit{Physical Review} style requires that the initial citation of
figures or tables be in numerical order in text, so don't cite
Fig.~\ref{fig:wide} until Fig.~\ref{fig:epsart} has been cited.



\end{comment}

\input acknowledgement.tex   % input acknowledgement

\begin{thebibliography}{99}

  \bibitem{LArIATDet}
    Standard LArIAT detector reference:  \\
R. Acciarri {\sl et al.} (LArIAT Collaboration),
hopefully we'll have one soon.

 \end{thebibliography}

\end{document}
%
% ****** End of file template.aps ******
